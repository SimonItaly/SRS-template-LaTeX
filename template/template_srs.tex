%%%%%%%%%%%%%%%%%%%%%%%%%%%%%%%%%%%%%%%%%%%%%%%%%%%%%%%%%%%%%%%%%%%%%
%
%	Copyright 2014 Jean-Philippe Eisenbarth
%		Original template from:
%		https://github.com/Eisenbarth/SRS-Tex
%
%	Copyright 2016 Simone Bisi
%		Riadattato per il corso "Progetto del Software"
%		presso Università degli studi di Modena e Reggio Emilia
%
%		Progetto sviluppato entro 15gg dal 16 novembre 2016
%
%%%%%%%%%%%%%%%%%%%%%%%%%%%%%%%%%%%%%%%%%%%%%%%%%%%%%%%%%%%%%%%%%%%%%

\documentclass{scrreprt}

%%%%%%%%%%%%%%%%%%%%%%%%%%%%%%%%%%%%%%%%%%%%%%%%%%%%%%%%%%%%%%%%%%%%%

\usepackage{listings}
\usepackage{underscore}
\usepackage[bookmarks=true]{hyperref}
\usepackage[utf8]{inputenc}
\usepackage[italian]{babel}
\usepackage{afterpage}

\usepackage{makeidx}
\usepackage{hyperref}
\usepackage{bookmark}

\usepackage{placeins}
\usepackage{multirow}

\usepackage{graphicx}

\makeindex

%%%%%%%%%%%%%%%%%%%%%%%%%%%%%%%%%%%%%%%%%%%%%%%%%%%%%%%%%%%%%%%%%%%%%

\def \author 	{Simone Bisi}
\def \project 	{Lorem ipsum dolor sit amet}
\def \from		{Università degli studi di Modena\\e Reggio Emilia}

\hypersetup{
    bookmarks=false,    								% bookmarks
    pdftitle={Specifica dei requisiti del software},    % title
    pdfauthor={\author},	                     		% author
    pdfsubject={Gummi 0.6.5 and LaTeX},    				% subject
    pdfkeywords={}, 	% list of keywords
    colorlinks=true,    % false: boxed links; true: colored links
    linkcolor=blue,     % color of internal links
    citecolor=black,    % color of links to bibliography
    filecolor=black,    % color of file links
    urlcolor=purple,    % color of external links
    linktoc=page        % only page is linked
}
\date{}

%%%%%%%%%%%%%%%%%%%%%%%%%%%%%%%%%%%%%%%%%%%%%%%%%%%%%%%%%%%%%%%%%%%%%

\begin{document}

\begin{flushright}
    \rule{16cm}{5pt}\vskip1cm
    \begin{bfseries}
        \Huge{SPECIFICA DEI REQUISITI\\ DEL SOFTWARE}\\
        \vspace{1.9cm}
        per\\
        \vspace{1.9cm}
        {\normalfont{\textit{\project}}}
        \vspace{1.9cm}
        \LARGE{\textsc{ }}\\
        \vspace{1.9cm}
        di \author\\
        \vspace{0.5cm}
        \textsc{\from}\\
        \vspace{2.0cm}
        \today\\
    \end{bfseries}
\end{flushright}

%%%%%%%%%%%%%%%%%%%%%%%%%%%%%%%%%%%%%%%%%%%%%%%%%%%%%%%%%%%%%%%%%%%%%

\newcommand\blankpage{%
    \null
    \thispagestyle{empty}%
    \addtocounter{page}{-1}%
    \newpage}
    
\blankpage{}
\blankpage{}

%%%%%%%%%%%%%%%%%%%%%%%%%%%%%%%%%%%%%%%%%%%%%%%%%%%%%%%%%%%%%%%%%%%%%

\tableofcontents

%%%%%%%%%%%%%%%%%%%%%%%%%%%%%%%%%%%%%%%%%%%%%%%%%%%%%%%%%%%%%%%%%%%%%

\chapter{Introduzione}
%Introduzione globale all'intero SRS
La presente sezione ha lo scopo di riportare la visione globale
dell'intero documento di specifica dei requisiti.
La struttura del documento è quella suggerita dallo standard
ANSI/IEEE 830 noto come SRS (Software Requirements Spcifications).

%====================================================================

\section{Obiettivo}
%Obiettivo del documento
%Utenza a cui è diretto
Lorem ipsum dolor sit amet.

%====================================================================

\section{Campo d'applicazione}
%Nome del prodotto da sviluppare
%Obiettivi del prodotto
%Principali benefici
%Problematiche che si intendono analizzare
%Problematiche che non saranno incluse nel processo di analisi
Lorem ipsum dolor sit amet.

%====================================================================

\section{Definizioni, acronimi, abbreviazioni}
%Definizione di tutti i termini utilizzati, acronimi e abbreviazioni

\FloatBarrier
\begin{table}[h|]
\centering
\begin{tabular}{|l|p{12cm}|}
\hline
\textbf{Entità} & \textbf{Definizione} \\ \hline
\textit{lorem}  & { lorem ipsum dolor sit amet } \\ \hline
\end{tabular}
\end{table}
\FloatBarrier

%====================================================================

\section{Fonti}
%Elenco completo di tutte le fonti (documenti, libri, siti, altro)
Lorem ipsum dolor sit amet.

%====================================================================

\pagebreak

\section{Struttura del documento}
%Organizzazione del documento delle specifiche
Il documento è diviso in varie sezioni, già specificate nell'indice
di testa.

\FloatBarrier
\begin{table}[h|]
\centering
\begin{tabular}{|l|l|p{10cm}|}
\hline
 & \textbf{Sezione} & \textbf{Funzione} \\ \hline

1 & \textit{Introduzione} & Presenta il prodotto soggetto alla
specifica dei requisiti e si illustrano eventuali termini
ricorrenti \\ \hline

2 & \textit{Descrizione generale} & Si cerca di inquadrare il
più possibile gli utilizzatori finali del software \\ \hline

3 & \textit{Specifica dei requisiti} & Espone i requisiti funzionali
e non funzionali necessari ad implementare il software \\ \hline

4 & \textit{Diagrammi} & Si mostrano i diagrammi dei casi d'uso,
delle attività e delle classi \\ \hline

5 & \textit{Design Patterns} & Si individuano design patterns per
facilitare la fase di progettazione del software \\ \hline
\end{tabular}
\end{table}
\FloatBarrier

%%%%%%%%%%%%%%%%%%%%%%%%%%%%%%%%%%%%%%%%%%%%%%%%%%%%%%%%%%%%%%%%%%%%%
%%%%%%%%%%%%%%%%%%%%%%%%%%%%%%%%%%%%%%%%%%%%%%%%%%%%%%%%%%%%%%%%%%%%%
%%%%%%%%%%%%%%%%%%%%%%%%%%%%%%%%%%%%%%%%%%%%%%%%%%%%%%%%%%%%%%%%%%%%%

\chapter{Descrizione generale}
%Principali fattori che riguardano il prodotto e i suoi requisiti
%Regole fondamentali per il concreto funzionamento del prodotto e per
%il soddisfacimento dei requisiti (sezione 3)
%Non ci sono ancora requisiti veri e propri
Questa sezione evidenzia i principali fattori che riguardano il
prodotto ed i requisiti fondamentali per il suo concreto funzionamento.

%====================================================================

\section{Inquadramento}
\label{sec:inquadramento}
%Confronto del sistema con altri prodotti simili
Lorem ipsum dolor sit amet.

%--------------------------------------------------------------------

	\subsection{Interfaccia di sistema}
	%Caratteristiche dell'interfaccia con il sistema
	Lorem ipsum dolor sit amet.

%--------------------------------------------------------------------

	\subsection{Interfaccia utente}
	%Caratteristiche dell'interfaccia con l'utente
	%In termini di formato dello screen, layout di pagine, contenuti
	%del report o dei menù, lunghezza messaggi di errore
	Lorem ipsum dolor sit amet.

%--------------------------------------------------------------------

	\subsection{Interfaccia hardware}
	%Informazioni per configurare adeguatamente il sistema
	Lorem ipsum dolor sit amet.

%--------------------------------------------------------------------

	\subsection{Interfaccia software}
	%Eventuale necessità d'uso di altri pacchetti software di 
	%supporto
	%Interlacciamento con altre applicazioni
	Lorem ipsum dolor sit amet.

%--------------------------------------------------------------------

	\subsection{Interfaccia di comunicazione}
	%Protocolli di comunicazione (TCP/IP)
	Lorem ipsum dolor sit amet.

%--------------------------------------------------------------------

	\subsection{Vincoli di occupazione di memoria}
	%Caratteristiche e limiti dei supporti di memoria
	%primaria e secondaria
	Lorem ipsum dolor sit amet.

%--------------------------------------------------------------------

	\subsection{Operazioni}
	%Operazioni di inizializzazione, backup e recovery del sistema
	Lorem ipsum dolor sit amet.

%--------------------------------------------------------------------

	\subsection{Vincoli di installazione}
	%Vincoli per ciascun nodo (es. sequenza inizializzazione, livello
	%di sicurezza richiesto, ...)
	Lorem ipsum dolor sit amet.

%====================================================================

\section{Funzioni del prodotto}
%Principali funzionalità, elencate a grandi linee
Lorem ipsum dolor sit amet.

%====================================================================

\section{Caratteristiche degli utenti finali}
%Caratteristiche degli utenti del sistema in termini di esperienza,
%capacità tecnica e livello di istruzione
Lorem ipsum dolor sit amet.

%====================================================================

\section{Vincoli generali}
%Vincoli che verranno affrontati durante lo sviluppo (interfacciamento
%con altri sistemi, operazioni parallele, elementi di criticità)
Lorem ipsum dolor sit amet.

%====================================================================

\section{Ipotesi iniziali}
%Ipotesi di partenza, assunzioni e dipendenze
%Fattori che, eventualmente modificati, hanno ripercussioni sul 
%contenuto dell'SRS
Lorem ipsum dolor sit amet.

%====================================================================

\section{Requisiti futuri}
%Requisiti che verranno dettagliati in futuro (meglio specificare
%quando e da chi)
Lorem ipsum dolor sit amet.

%%%%%%%%%%%%%%%%%%%%%%%%%%%%%%%%%%%%%%%%%%%%%%%%%%%%%%%%%%%%%%%%%%%%%
%%%%%%%%%%%%%%%%%%%%%%%%%%%%%%%%%%%%%%%%%%%%%%%%%%%%%%%%%%%%%%%%%%%%%
%%%%%%%%%%%%%%%%%%%%%%%%%%%%%%%%%%%%%%%%%%%%%%%%%%%%%%%%%%%%%%%%%%%%%

\chapter{Specifica dei requisiti}
%Sezione principale, riporta i requisiti funzionali e non funzionali
La sezione di specifica dei requisiti è la parte principale del
documento e riporta tutti i requisiti del prodotto, divisi nelle
categorie \textit{funzionali} e \textit{non funzionali}.

%====================================================================

\section{Requisiti dell'interfaccia esterna}
%Specifica l'interfaccia del software verso l'esterno (I/O)
%Complementare alle specifiche in 2.1 (Inquadramento)
Lorem ipsum dolor sit amet.

%====================================================================

\section{Requisiti funzionali}
%Ogni requisito è descritto tramite una scheda/tabella
%
%	- introduzione
%		> attori coinvolti
%		> descrizione generale della funzione
%	- input
%		(informazioni che sono gestite all'interno del processo
%		come ingresso)
%		> descrizione generica dei dati
%	- descrizione (processo)
%		(sequenza di azioni eseguite dall'operatore, eventuali
%		messaggi di errore come risposta ad anomalie e parametri
%		che incidono sull'output)
%		> validazione dei dati
%		> sequenza di operazioni
%		> risposta ad eventuali anomalie
%		> parametri che impattano sull'output
%	- output
%		(risultato del processo)
%
%	(tabella alternativa)
%	----------------------------------------------------
%	Codice 		| Area di riferimento | Titolo specifico
%	----------------------------------------------------
%	Input  		|
%	----------------------------------------------------
%	Descrizione |
%	(processo)  |
%	----------------------------------------------------
%	Output		|
%	----------------------------------------------------
%

\newcounter{rfu}
\stepcounter{rfu}

I requisiti funzionali descrivono come il sistema deve comportarsi in
base agli input degli utenti finali e nelle varie situazioni d'uso.

\newcommand{\specialcell}[2][c]{%
  \begin{tabular}[#1]{@{}l@{}}#2\end{tabular}}

%... Visualizza stato
\begin{table}[h|]
\centering
\begin{tabular}{|l|p{6cm}|p{6cm}|}
\hline
\textbf{RF\therfu} & \textbf{-} & \textbf{-} \\ \hline
Attori 		& \multicolumn{2}{p{12cm}|}{ - } \\ \hline
Input  		& \multicolumn{2}{p{12cm}|}{ \specialcell{-\\-} }   \\ \hline
Descrizione & \multicolumn{2}{p{12cm}|}{-}               		\\ \hline
Output  	& \multicolumn{2}{p{12cm}|}{ \specialcell{-\\-} }	\\ \hline
\end{tabular}
\end{table}
\stepcounter{rfu}

%====================================================================

\section{Requisiti non funzionali}
I requisiti non funzionali descrivono le proprietà e le qualità dei
servizi offerti dal sistema in maniera quantificabile.

\newcounter{rnf}
\stepcounter{rnf}

	\subsection{Requisiti prestazionali}
	%Numero di terminali supportati
	%Numero di utenti che hanno accesso al sistema contemporaneamente
	%Quantità e tipo di informazioni che possono essere manipolate
	%contemporaneamente
	
%--------------------------------------------------------------------
	
	\FloatBarrier
	\begin{table}[h|]
	\centering
	\begin{tabular}{|l|p{6cm}|p{6cm}|}
	\hline
	\textbf{RNF\thernf} & \textbf{Requisiti prestazionali} & \textbf{Tempi di risposta} \\ \hline
	Descrizione  & \multicolumn{2}{p{12cm}|}{ Lorem ipsum dolor sit amet. }             \\ \hline
	\end{tabular}
	\end{table}
	\FloatBarrier
	\stepcounter{rnf}

%--------------------------------------------------------------------

	\subsection{Database}
	%Tipo di database che si intende utilizzare
	%Database non compatibili con le caratteristiche del prodotto
	\FloatBarrier
		\begin{table}[h|]
	\centering
	\begin{tabular}{|l|p{6cm}|p{6cm}|}
	\hline
	\textbf{RNF\thernf} & \textbf{Database} & \textbf{Gestione dei dati} \\ \hline
	Descrizione  & \multicolumn{2}{p{12cm}|}{Lorem ipsum dolor sit amet.}\\ \hline
	\end{tabular}
	\end{table}
	\FloatBarrier
	\stepcounter{rnf}

%--------------------------------------------------------------------

	\subsection{Vincoli generali di progetto}
	%Vincoli imposti da altri standard
	%Limitazioni hardware
	%Standard adottati
	%	requisiti che derivano dagli standard esistenti
	%	requisiti che derivano delle norme vigenti
	%Altro
	\FloatBarrier
	\begin{table}[h|]
	\centering
	\begin{tabular}{|l|p{6cm}|p{6cm}|}
	\hline
	\textbf{RNF\thernf} & \textbf{Vincoli generali} & \textbf{-} \\ \hline
	Descrizione  & \multicolumn{2}{p{12cm}|}{ Lorem ipsum dolor sit amet.} \\ \hline
	\end{tabular}
	\end{table}
	\FloatBarrier
	\stepcounter{rnf}

%--------------------------------------------------------------------
	
	\subsection{Attributi del sistema}
	\addtokomafont{labelinglabel}{\sffamily}
	\begin{labeling}{asd}
	\item
	\begin{labeling}{Accessibilità del sistema}
		
		\item [Affidabilità]
			%È necessario specificare le condizioni che 
			%determineranno il grado di affidabilità accettabile del 
			%software al momento del rilascio
			Lorem ipsum dolor sit amet.
		
		\item [Accessibilità del sistema]
			%Elenco di parametri che determinano l'accessibilità
			%dell'intero prodotto software in termini di checkpoint,
			%le funzionalità di recovery ed altri parametri di
			%accessibilità
			Lorem ipsum dolor sit amet.
		
		\item [Sicurezza]
			%Elenco delle proprietà che servono a proteggere il
			%software dagli accessi accidentali o di utenti 
			%malintenzionati
			Lorem ipsum dolor sit amet.
		
		\item [Mantenibilità]
			%Elenco delle proprietà che rendono il prodotto software
			%mantenibile, ad esempio la modularità, software 
			%organizzato a moduli, elementi critici paramretrizzati
			Lorem ipsum dolor sit amet.
		
		\item [Scalabilità]
			Lorem ipsum dolor sit amet.
		
		\item [Portabilità]
			%Elenco delle proprietà che rendono il prodotto software,
			%ad esempio, indipendente dal sistema operativo, dal 
			%DBMS, e simili
			Lorem ipsum dolor sit amet.
	\end{labeling}
	\end{labeling}

%--------------------------------------------------------------------

	\subsection{Altri requisiti}
	%Elenco dei requisiti che descrivono (ad esempio) funzionamento
	%del software in modalità sperimentale ed a regime, gruppi di 
	%utenti e relativi permessi per l'accesso alle funzionalità del 
	%sistema, commenti addizionali, scalabilità, aumento del numero 
	%di utenti, implicazione sulla performance, costi
	Lorem ipsum dolor sit amet.

%%%%%%%%%%%%%%%%%%%%%%%%%%%%%%%%%%%%%%%%%%%%%%%%%%%%%%%%%%%%%%%%%%%%%
%%%%%%%%%%%%%%%%%%%%%%%%%%%%%%%%%%%%%%%%%%%%%%%%%%%%%%%%%%%%%%%%%%%%%
%%%%%%%%%%%%%%%%%%%%%%%%%%%%%%%%%%%%%%%%%%%%%%%%%%%%%%%%%%%%%%%%%%%%%

\chapter{Diagrammi}

\section{Diagramma dei casi d'uso}
I diagrammi \textit{Use Case} sono dedicati alla descrizione delle
funzioni o servizi offerti da un sistema, così come sono percepiti e
utilizzati dagli attori che interagiscono col sistema stesso.

	\FloatBarrier
	\begin{table}[h|]
	\begin{tabular}{|c|}
	\hline
	{ . } \\
	\hline
	\end{tabular}
	\end{table}
	\FloatBarrier

	\FloatBarrier
	\begin{table}[h|]
	\centering
	\begin{tabular}{p{3cm}p{11cm}}
	\textbf{Use case} & \textit{Nome} \\ 
	\textbf{Attori} & - (iniziatore), - \\ 
	\textbf{Tipo} & - \\ 
	\textbf{Descrizione} & - \\
	\\
	\end{tabular}
	\centering
	\begin{tabular}{|p{7cm}|p{7cm}|}
	\hline
	\textbf{Azione attore} & \textbf{Risposta del sistema} \\ \hline
	1. (azione attore) &                  \\ \hline
	& 2. (risposta sistema) \\ \hline
	\multicolumn{2}{|c|}{\textbf{Eccezioni}} \\ \hline
	\multicolumn{2}{|l|}{ 2. (errore sistema) } \\ \hline
	\end{tabular}
	\end{table}
	\FloatBarrier

%====================================================================		

\pagebreak

\section{Diagramma delle attività}
	\FloatBarrier
	\begin{table}[h|]
	\begin{tabular}{|c|}
	\hline
	{ . } \\
	\hline
	\end{tabular}
	\end{table}
	\FloatBarrier

%====================================================================

\pagebreak

\section{Diagramma delle classi}
	\FloatBarrier
	\begin{table}[h|]
	\begin{tabular}{|c|}
	\hline
	{ } \\
	\hline
	\end{tabular}
	\end{table}
	\FloatBarrier

%====================================================================

\pagebreak

\section{Diagramma di sequenza}
	\FloatBarrier
	\begin{table}[h|]
	\begin{tabular}{|c|}
	\hline
	{ . } \\
	\hline
	\end{tabular}
	\end{table}
	\FloatBarrier

%%%%%%%%%%%%%%%%%%%%%%%%%%%%%%%%%%%%%%%%%%%%%%%%%%%%%%%%%%%%%%%%%%%%%
%%%%%%%%%%%%%%%%%%%%%%%%%%%%%%%%%%%%%%%%%%%%%%%%%%%%%%%%%%%%%%%%%%%%%
%%%%%%%%%%%%%%%%%%%%%%%%%%%%%%%%%%%%%%%%%%%%%%%%%%%%%%%%%%%%%%%%%%%%%

\chapter{Design Patterns}

I seguenti design pattern rappresentano soluzioni riusabili per
trattare problemi ricorrenti nella realizzazione del prodotto.\\
Sono indipendenti dal linguaggio di programmazione utilizzato.

%====================================================================

\section{Classe pattern}
Lorem ipsum dolor sit amet.

%--------------------------------------------------------------------

\subsection{Nome pattern}
Lorem ipsum dolor sit amet.

%%%%%%%%%%%%%%%%%%%%%%%%%%%%%%%%%%%%%%%%%%%%%%%%%%%%%%%%%%%%%%%%%%%%%
%%%%%%%%%%%%%%%%%%%%%%%%%%%%%%%%%%%%%%%%%%%%%%%%%%%%%%%%%%%%%%%%%%%%%
%%%%%%%%%%%%%%%%%%%%%%%%%%%%%%%%%%%%%%%%%%%%%%%%%%%%%%%%%%%%%%%%%%%%%

%\chapter{Appendici}
%Tutto ciò che può essere utile ma non è essenziale per specificare
%i requisiti (norme, convenzioni, supporti)
%Lorem ipsum dolor sit amet.

%%%%%%%%%%%%%%%%%%%%%%%%%%%%%%%%%%%%%%%%%%%%%%%%%%%%%%%%%%%%%%%%%%%%%
%%%%%%%%%%%%%%%%%%%%%%%%%%%%%%%%%%%%%%%%%%%%%%%%%%%%%%%%%%%%%%%%%%%%%
%%%%%%%%%%%%%%%%%%%%%%%%%%%%%%%%%%%%%%%%%%%%%%%%%%%%%%%%%%%%%%%%%%%%%

% Indice analitico
% Uso: Parola\index{parola}

%\printindex

%%%%%%%%%%%%%%%%%%%%%%%%%%%%%%%%%%%%%%%%%%%%%%%%%%%%%%%%%%%%%%%%%%%%%
%%%%%%%%%%%%%%%%%%%%%%%%%%%%%%%%%%%%%%%%%%%%%%%%%%%%%%%%%%%%%%%%%%%%%
%%%%%%%%%%%%%%%%%%%%%%%%%%%%%%%%%%%%%%%%%%%%%%%%%%%%%%%%%%%%%%%%%%%%%

\pagebreak
\blankpage{}

\end{document}
